%% Le lingue utilizzate, che verranno passate come opzioni al pacchetto babel. Come sempre, l'ultima indicata sar� quella primaria.
%% Se si utilizzano una o pi� lingue diverse da "italian" o "english", leggere le istruzioni in fondo.
\def\thudbabelopt{english,italian}
%% Valori ammessi per target: bach (tesi triennale), mst (tesi magistrale), phd (tesi di dottorato).
\documentclass[target=mst]{thud}[2014/03/11]

%% Aggiunta pacchetto per inserimento di immagini
\usepackage{graphicx}
\graphicspath{ {img/Introduzione/} {img/ContestoTecnologico/1_PCNEurotech/} {img/ContestoTecnologico/2_TecnTesi/} {img/CorpoTesi/1_AmbientiLavoro/} }

%% --- Informazioni sulla tesi ---
%% Per tutti i tipi di tesi
\title{Elaborazione di Immagini in ambito Embedded con OpenCV: Passenger Counter}
\author{Mattia Dal Ben}
\course{Ingegneria Elettronica}
\supervisor{Prof.\ Antonio Abramo}
\cosupervisor{Ing.\ Marco Carrer}
%% Altri campi disponibili: \reviewer, \tutor, \chair, \date (anno accademico, calcolato in automatico).
%% Con \supervisor, \cosupervisor, \reviewer e \tutor si possono indicare pi� nomi separati da \and.
%% Per le sole tesi di dottorato
\phdnumber{313}
\cycle{XXVIII}
\contacts{Via della Sintassi Astratta, 0/1\\65536 Gigatera --- Italia\\+39 0123 456789\\\texttt{http://www.example.com}\\\texttt{inbox@example.com}}
\rights{Tutti i diritti riservati a me stesso e basta.}
%% Campi obbligatori: \title, \author e \course.

%% --- Pacchetti consigliati ---
%% hyperref: Regola le impostazioni della creazione del PDF... pi� tante altre cose.
%% tocbibind: Inserisce nell'indice anche la lista delle figure, la bibliografia, ecc.

%% --- Stili di pagina disponibili (comando \pagestyle) ---
%% sfbig (predefinito): Apertura delle parti e dei capitoli col numero grande; titoli delle parti e dei capitoli e intestazioni di pagina in sans serif.
%% big: Come "sfbig", solo serif.
%% plain: Apertura delle parti e dei capitoli tradizionali di LaTeX; intestazioni di pagina come "big".

\begin{document}

%% Il frontespizio prima di tutto!
\maketitle

%% Dedica (opzionale)
%% \begin{dedication}A mia madre.\end{dedication}

%% Ringraziamenti (opzionali)
%% \acknowledgements
%% Sed vel lorem a arcu faucibus aliquet eu semper tortor. Aliquam dolor lacus, semper vitae ligula sed, blandit iaculis leo. Nam pharetra lobortis leo nec auctor. Pellentesque habitant morbi tristique senectus et netus et malesuada fames ac turpis egestas. Fusce ac risus pulvinar, congue eros non, interdum metus. Mauris tincidunt neque et aliquam imperdiet. Aenean ac tellus id nibh pellentesque pulvinar ut eu lacus. Proin tempor facilisis tortor, et hendrerit purus commodo laoreet. Quisque sed augue id ligula consectetur adipiscing. Vestibulum libero metus, lacinia ac vestibulum eu, varius non arcu. Nam et gravida velit.

%% Sommario (opzionale)
\abstract
Nello sviluppo di questa tesi si \`e affrontato lo studio e la progettazione di un sistema di conteggio dei passeggeri su una piattaforma embedded. Il software applicativo \`e basato su algoritmi di elaborazione di immagine resi disponibili dalla libreria open-source per l'image processing OpenCV. La piattaforma software \`e stata realizzata utilizzando il progetto Yocto, il quale permette la creazione di distribuzioni Linux customizzate e targettate all'utilizzo in abito embedded.
L'applicazione di Passenger Counter ha come scopo quello di contare i passeggeri che attraversano in entrata e in uscita le porte di un mezzo pubblico, in modo tale da permettere un conteggio esatto delle persone presenti sul mezzo.
Lo sviluppo si \`e diviso in quattro fasi principali:
\begin{itemize}
\item Una indagine preliminare sulle migliori piattaforme sulle quali sviluppare l'applicazione.
\item Progettazione e implementazione del contatore usando solamente algoritmi di elaborazione delle immagini (Passenger Counter con background subtraction), individuando i passaggi pi\`u pesanti dal punto di vista computazionale.
\item Progettazione e implementazione del contatore sfruttando telecamere a infrarossi Time-Of-Flight (Passenger counter con telecamere RealSense).
\item Lo sviluppo della piattaforma software sulla quale integrare tutte le tecnologie utilizzate in fase di sviluppo per mezzo del progetto Yocto.
\end{itemize}

%% Indice
\tableofcontents

%% Lista delle tabelle (se presenti)
%\listoftables

%% Lista delle figure (se presenti)
\listoffigures

%% Corpo principale del documento
\mainmatter

%% Parte
%% La suddivisione in parti � opzionale; talvolta sono sufficienti i capitoli.
%% \part{Parte}

%% Capitolo 1: Introduzione
\chapter{Introduzione}
Nel seguito viene riportato il contesto all'interno del quale si configura l'applicazione sviluppata nel corso della tesi. Quindi verr\`a descritto l'applicazione del Passenger Counter e gli obiettivi di questa tesi. Nel secondo capitolo verr\`a affrontato il contesto tecnologico di dettaglio mentre nel terzo capitolo verr\`a trattata la realizzazione vera e propria dell'applicazione. 

%% Sezione Contesto IoT
%% NOTA: L'intera sezione che segue \`e stata presa dalla Wikipedia inglese alla voce: Internet of Things.
%% TODO: Cambiare un po' le parole in modo che non si capisca che ho copiato =D
\section{Il contesto IoT}
L'Internet of Things (IoT) \`e l'internconnesione di device, veicoli, edifici e oggetti dotati di elettronica, software, sensori, attuatori e connettivit\`a che permettono a questi oggetti di raccogliere e scambiare dati. L'IoT permette agli oggetti di essere rilevati e controllati in remoto attraverso l'infrastruttura di rete esistente, creando opportunit\`a per una integrazione pi\`u diretta del mondo fisico all'interno di sistemi informatizzati, con l'obbiettivo di aumentare l'efficienza, la precisione e il beneficio economico riducendo al contempo la necesssit\`a dell'intervento umano. Tipicamente ci si aspetta che l'IoT offra connettivit\`a avanzata tra device, sistemi e servizi che vadano oltre la comunicazione Machine-to-machine (M2M) e coprano una variet\`a di protocolli, domini ed applicazioni. L'obiettivo \`e quello di introdurre processi di automazione in tutti i settori.

%% Storia dell'IoT
\subsection{Storia dell'Internet of Things}
Il neologismo inglese Internet of Things \`e stato introdotto per la prima volta da Kevin Ashton, cofondatore e direttore esecutivo di Auto-ID Center (consorzio di ricerca con sede al MIT), durante una presentazione nel 1999, ma il concetto di una rete di device "intelligenti" fu discusso per la prima volta nel 1982, con un distributore di bibite opportunamente modificato per interfacciarsi ad internet dalla Carniegie Mellon University. Esso era capace di riportare il suo inventario e qualora le bibite di cui era appena stato rifornito fossero ancora calde. Tra il 1993 e il 1996 molte aziende cominciarono a proporre soluzioni per l'Internet delle Cose ma \`e solamente dopo il 1999 che il settore cominci\`o ad assumere rilevanza. Le prime applicazioni per questo tipo di concetti erano l'inventariamento degli oggetti all'interno di fabbriche. Ci\`o poteva essere realizzato utilizzando tag RFID (Radio-frequency Identification) che permettessero ai sistemi informatici di identificare e tracciare gli oggetti presenti all'interno di ambienti vasti. Questo tipo di applicazione dell'IoT \`e ormai pratica standard nota come RFID Asset Tracking.
Ad oggi il concetto di IoT si \`e molto evoluto grazie al progresso tecnologico. La capacit\`a di integrare negli oggetti elettronica, sensori e connettivit\`a wireless ne ha ampliato le capacit\`a e le possibili applicazioni. Nel settore IoT ora convergono molteplici tencologie quali real-time analytics, machine learning, commodity sensors e sistemi embedded.

%% Parte di questa sezione l'ho rubacchiata da wikipedia cercando: Industria 4.0
%% TODO: Cambiare un po' le parole in modo che non si capisca che ho copiato =D
\subsection{Sviluppi futuri}
\begin{figure}[h!]
  \includegraphics[width=\linewidth]{Internet_of_Things.png}
  \caption{Trend dell'internet of things}
  \centering
  \label{fig:trendIoT}
\end{figure}
Secondo le proiezioni di Gartner, Inc. una corporation per la ricerca e advisory tecnologica, entro il 2020 ci saranno oltre 20 miliardi di device connesse all'Internet of Things. Si sta parlando di una Industria 4.0 dove l'automazione industriale integra l'IoT per migliorare le condizioni di lavoro e la produttivit\`a. La chiave di volta dell'Industria 4.0 sono i sistemi ciberfisici (CPS), ovvero sistemi fisici che sono strettamente connessi con i sistemi informatici e possono interagire e collaborare con altri sistemi CPS i quali costituiscono lo step evolutivo successivo delle device IoT. 
Un altro settore nel quale si proiettano ulteriori sviluppi \`e la Big Data Analysis. L'ubiquit\`a dei dispositivi intelligenti connessi all'Internet delle Cose permette analisi di dati vastissimi ai quali precedentemente era impensabile avere accesso. Le informazioni che si possono ricavare da questi dati sono molteplici e di sicuro interesse per molti ambiti di applicazione.

%% Interruzione di pagina
\newpage

%% Classi di applicazioni IoT
\section{Classi di applicazioni IoT}
Passiamo ora ad analizzare i campi di applicazione pi\`u diffusi per l'Internet delle Cose.

%% Qui ho rubato da Wikipedia inglese il paragrafo sulle Applications nella pagina dedicata all'IoT
%% TODO: Cambiare un po' le parole in modo che non si capisca che ho copiato =D
\subsection{Environmental monitoring}
Le applicazioni di monitoraggio ambientale tipicamente utilizzano sensori collegati all'Internet delle Cose per fornire assistenza nella protezione dell'ambiente. Il monitoraggio pu\`o interessare la qualit\`a dell'aria o dell'acqua, condizioni atmosferiche o del suolo, e pu\`o includere aree come il monitoraggio degli spostamenti della fauna selvatica e del suo habitat. Ci\`o pu\`o avere applicazioni anche nell'ambito della rilevazione di disastri naturali: sistemi di allerta per terremoti e tsunami possono essere implementati nell'ambito IoT. Device IoT in questo campo di applicazione sono tipicamente dislocate su un'ampia area geografica possono anche essere mobili.

\subsection{Infrastructure management}
Il monitoraggio e il controllo di infrastrutture urbane come ponti, rotaie, wind-farms \`e un campo di applicazione chiave dell'IoT. L'infrastruttura IoT pu\`o essere usata per monitorare eventi o cambiamenti nelle condizioni strutturali che possono compromettere la sicurezza o aumentare i rischi. Pu\`o altres\`\i\ essere usato per la programmazione di interventi di manutenzione in maniera pi\`u efficiente coordinando le operazioni tra diversi fornitori di servizi. Le device IoT sono anche usate per controllare infrastrutture critiche come ponti per fornire accesso alle navi. Si pensa che l'utilizzo di device IoT possa migliorare la gestione degli incidenti e la coordinazione in caso emergenze, la qualit\`a del servizia e ridurre i costi in tutte le aree legate alla gestione delle infrastrutture.

\subsection{Manufacturing}
L'ambiente manifatturiero \`e sicuramente una delle applicazioni di punta dell'IoT fina dalla sua nascita. Gli inteventi dell'IoT spaziano dalla gestione degli equipaggiamenti al tracking degli asset nell'ambiente produttivo. Questa sinergia permette alle aziende di avere una maggior flessibilit\`a e di ottimizzare in tempo reale i sistemi di produzione nonch\`e la rete di approviggionamento grazie all'interconnesione tra macchinari, sensori e sistemi.
L'integrazione di sistemi IoT all'interno di catene di produzione permette la predictive maintenance, valutazione statistica della degradazione dello stato dei macchinari, e prevenzione di guasti.
Come visto in precedenza l'evoluzione dell'Industria 4.0 \`e interamente basata sull'integrazione tra sistemi di produzione e l'Internet of Things.

\subsection{Energy management}
L'intregrazione di reti di sensori e attutatori, connessi ad internet, si pensa possa ottimizzare il consumo energetico in ambito industriale e casalingo. L'integrazione di device IoT all'interno di contatori per l'energia, cos\`\i\ come dispositivi domestici e industriali, capaci di comunicare con le compagnie per la rete elettrica possono permettere una gestione migliore della generazione ed utilizzo dell'energia. Questo tipo di device inoltre permetterebbe agli utenti di controllare in remoto i loro dispositivi, o controllarli centralmente grazie a interfacce cloud, in modo tale da implementare funzioni avanzate come la programmazione delle accensioni o dell'utilizzo dei dispositivi. Questo tipo di applicazioni \`e molto diffuso nel campo della domotica, i termostati IoT sono un ottimo esempio di questa applicazione.

\subsection{Medical and Healthcare}
Device connesse all'Internet delle cose possono essere abilitare il monitoraggio remoto dello stato di salute e sistemi di notifica delle emergenze. Queste device per il rilevamento dello stato di salute possono variare dal rilevamento della pressione arteriosa fino al conteggio dei battiti del cuore. Alcuni ospedali hanno cominciato ad utilizzare "letti smart" in grado di rilevare quanto il letto \`e occupato e quando il paziente sta cercando di alzarsi. Ormai sono molto diffusi dispositivi per il tracciamento delle attivit\`a dell'utente che incoraggiano uno stile di vita salutare, in questo ambito rientrano le wearable device come i fitness trackers.

\subsection{Settore dei trasporti}
%% Qui mi ricollego al PCN che \`e la sezione seguente
L'IoT pu\`o dare assistenza nella integrazione delle comunicazioni, controlli e elaborazione delle informazioni attraverso vari sistemi di trasporto. Le applicazioni dell'IoT si estendono a tutti gli aspetti dei sistemi di trasporto: i veicoli, l'infrastruttura, il pilota e i passeggeri. L'interazione dinamica tra questi componenti permette una comunicazione inter e intra veicolare, controllo del traffico intelligente, smart parking, gestione della logistica e delle flotte, controllo dei veicoli e sicurezza stradale.
L'applicazione sviluppata nel corso della tesi \`e appunto legata a questo ambito di applicazione dell'IoT e nel prossimo paragrafo ne analizzeremo gli obiettivi e funzionalit\`a.

%% Interruzione di pagina
\newpage

%% Sezione PCN
\section{Passenger Counter}
Il Passenger Counter, o contatore di passeggeri, \`e un dispositivo IoT la cui funzione \`e quella di rilevare e conteggiare i passeggeri presenti all'interno di un sistema di trasporto pubblico. Esso deve altres\`\i\ fornire i dati in tempo reale al gestore del servizio, sfruttando l'Internet delle Cose, in modo tale che sia possibile:
\begin{itemize}
\item Rilevare frodi.
\item Aumentare l'efficienza della flotta di mezzi migliorando la gestione e la programmazione dei percorsi.
\item Restringere il numero di persone sul mezzo per ragioni di sicurezza.
\item Analizzare i flussi di traffico all'interno delle citt\`a.
\end{itemize}
Il dispositivo deve essere in grado di effettuare il conteggio in modo non invasivo e contact-less, tenendo conto delle restrizioni dovute all'ambiente nel quale deve essere applicato.

\begin{figure}[h!]
  \includegraphics[width=9cm]{PassengerCountersz.jpg}
  \centering
  \caption{Schematizzazione del contatore di passeggeri}
  \label{fig:SchemPCN}
\end{figure}

Il lavoro presentato in questa tesi \`e stato commissionato da Eurotech: azienda dedicata alla ricerca, sviluppo e produzione di sistemi embedded e computer ad alte prestazioni e con sede ad Amaro. L'azienda inoltre ha una forte rilevanza in ambito IoT in quanto, oltre a fornire dispositivi IoT ready, ha realizzato una piattaforma Machine-to-Machine che consente ai dispositivi, ai sensori e a tutti i sistemi distribuiti sul campo di comunicare tra loro trasferendo le informazioni rilevanti alle business application ed alle infrastrutture IT. Il PCN o Passenger Counter \`e uno dei prodotti di punta dell'azienda ed \`e stato selezionato come progetto per questa tesi.

%% Interruzione di pagina
\newpage

\begin{figure}[h!]
  \includegraphics[width=\textwidth]{SchemaPCNIoT.png}
  \centering
  \caption{Schema Passenger Counter Eurotech}
  \label{fig:SchemPCNEurotech}
\end{figure}

Il sistema di conteggio di passeggeri consta di tre parti fondamentali come schematizzato in figura \ref{fig:SchemPCNEurotech}:
\begin{enumerate}
\item Dispositivo di acquisizione video.
\item Dispositivo di elaborazione e trasmissione dei dati in real-time.
\item Infrastruttura di rete cloud per il raccoglimento dei dati.
\end{enumerate}
\`E quindi possibile accedere ai dati raccolti da pi\`u dispositivi per mezzo della piattaforma M2M proprietaria di Eurotech.

\section{Obiettivi della tesi}
L'obiettivo della tesi \`e stato quello di realizzare una nuova versione del Passenger Counter di Eurotech basandosi sulla infrastruttura software/hardware fornita dall'azienda, cercando di migliorare quanto gi\`a fatto da Eurotech, esplorando soluzioni tecnologiche alternative. Questo obiettivo \`e stato quindi suddiviso in tre parti:
\begin{enumerate}
\item Identificare gli ambienti di sviluppo pi\`u idonei alla realizzazione del progetto.
\item Realizzare una nuova versione del Passenger Counter puntando a migliorare le prestazione ed abbattere i costi, adattandolo alle tecnologie usate dall'azienda per i suoi prodotti. 
\item Realizzare una infrastruttura software che fornisse tutti gli strumenti e le librerie necessarie all'implementazione del Passenger Counter realizzato e che potesse essere installato sulla piattaforma hardware fornita da Eurotech.
\end{enumerate}

%% Capitolo 2: Contesto tecnologico di dettaglio
\chapter{Contesto tecnologico di dettaglio}
Passiamo ora a descrivere pi\`u tecnicamente e nel dettaglio le tecnologie utilizzare nello svolgimento della tesi.

\section{Tecnologie PCN Eurotech}

\begin{figure}[h!]
  \includegraphics[width=\textwidth]{StrutturaPCN.png}
  \centering
  \caption{Struttura del Passenger Counter di Eurotech}
  \label{fig:PCNEurotechHardware}
\end{figure}

%% Interruzione di pagina
\newpage

\subsection{Hardware}
Il sistema di conteggio dei passeggeri consta di due componenti hardware fondamentali: il gateway e il dispositivo di acquisizione delle immagini.

Il gateway \`e un'altro prodotto della Eurotech noto come ReliGATE 50-21. Utilizza un processore x86 Intel Atom Z510P al quale sono state aggiunte opportune interfacce di rete per il deployment mobile. Esso ha infatti interfacce 2G/3G, WiFi, 802.15.4/Zigbee e GPS. Questo componente rappresenta l'unit\`a di elaborazione centrale del sistema nonch\`e il mezzo attraverso il quale i dati di conteggio vengono raccolti e trasmessi alla piattaforma cloud della Eurotech.

\begin{figure}[h!]
  \includegraphics[width=\textwidth]{DispositivoAcquisizioneImmagini.png}
  \centering
  \caption{Dispositivo di acquisizione delle immagini 3D}
  \label{fig:DispAcqImm3D}
\end{figure}

Il dispositivo di acquisizione video visibile in figura \ref{fig:DispAcqImm3D}. Esso nasconde al suo interno una FPGA programmata con una IP proprietaria Eurotech che permette la ricostruzione in 3D delle immagini acquisite dalle telecamere stereoscopiche di cui \`e dotato il dispositivo.
Il funzionamento \`e il seguente: i proiettori di luce infrarosse illuminano la scena, le telecamere a infrarossi stereoscopiche acquisiscono le immagini nello spettro della luce infrarossa. Le due immagini vengono passate alla FPGA che per mezzo di tecniche di ricostruzione stereoscopica accelerate in hardware permette di generare delle immagini in scala di grigi dove l'informazione di profondit\`a \`e data dal colore del pixel. Queste immagini vengono quindi passate all'unit\`a di elaborazione centrale che vi applica un semplice algoritmo per il tracciamento dei passeggeri di cui discuteremo il funzionamento nel dettaglio nel prossimo paragrafo.

%% Interruzione di pagina
\newpage

\subsection{Algoritmo per il tracciamento dei passeggeri}

\begin{figure}[h!]
  \includegraphics[scale=0.75]{Algoritmov2.png}
  \centering
  \caption{Schematizzazione funzionamento algoritmo tracciamento passeggeri}
  \label{fig:SchemAlgrTrackPCNEurotech}
\end{figure}

%% TODO: Chiedere al prof se il conteggio viene effettuato sul ReliGate o sul DynaPCN
L'unit\`a di elaborazione centrale si vede arrivare in ingresso una immagine in scala di grigi in cui \`e codificata l'informazione di profondit\`a tramite il colore dei pixel. I punti pi\`u vicini al rilevatore tendono al bianco, i punti pi\`u lontani tendono al nero.
Il cuore dell'algoritmo opera come segue:
\begin{enumerate}
\item Vengono rilevati i massimi locali all'interno dell'imagine in scala di grigi, i quali rappresentano le teste delle persone.
\item Traccia la posizione nel tempo di questi massimi locali all'interno del flusso di immagini.
\item Rileva quando questi massimi locali attraversano una linea di demarcazione virtuale, la quale indica l'ingresso o uscita dalla soglia e aggiorna i contatori.
\end{enumerate}
In figura \ref{fig:SchemAlgrTrackPCNEurotech} \`e riportata una schematizzazione di quanto accade durante il conteggio.

%% Interruzione di pagina
\newpage

\subsection{Infrastruttura software}

\begin{figure}[h!]
  \includegraphics[width=\textwidth]{SoftwareStack.png}
  \centering
  \caption{Schematizzazione stack software del sistema di Passenger Counting}
  \label{fig:StackSoftwarePCNEurotech}
\end{figure}

La piattaforma sulla quale si basa il Passenger Counter Eurotech \`e basata su una distribuzione custom di Linux realizzata per mezzo del progetto Yocto (Per affrondimenti circa questo strumento si veda il paragrafo \ref{Yocto}). Ad essa sono stati aggiunti i driver proprietari per le tecnologie Eurotech. Sopra di essa \`e presente la Java Virtual Machine la quale permette l'esecuzione del framework OSGi (Per affrondire questo framework si veda il paragrafo \ref{OSGi}). Su OSGi \`e basato il framework proprietario di Eurotech: Everyware Software Framework (ESF). Esso permette il deployment di applicazioni del cliente in modo flessibile tramite interfaccia grafica da web, la quale \`e resa disponibile dall'infrastruttura M2M di Eurotech.
Nei prossimi capitoli vedremo come sia stato necessario adattare l'applicazione realizzata nel corso di questa tesi alla infrastruttura qui descritta, nonch\`e modificare parte dell'infrastruttura per aggiungere funzionalit\`a mancanti e necessari alla nuova versione del Passenger Counter.

%% Interruzione di pagina
\newpage

\subsection{Problematiche di questa soluzione}
Questo sistema di conteggio dei passeggeri non \`e per\`o esente da problematiche, le quali sono state il punto di partenza per il mio lavoro. Le principali sono le seguenti:

\begin{enumerate}
\item Le ottiche del dispositivo di acquisizione delle immagini sono costose e difficili da reperire.
\item L'utilizzo di una FPGA per la ricostruzione delle informazioni di profondit\`a fa aumentare i costi di produzione.
\item Per come \`e stata implementata la soluzione \`e necessario installare una unit\`a computazionale per dispositivo di acquisizione delle immagini. Anche questo contribuisce ad aumentare i costi di produzione di un sistema PCN.
\end{enumerate}

Vedremo che nella versione del Passenger Counter realizzata nel corso di questa tesi sono stati risolti in buona parte tutte queste problematiche, con vari gradi di successo.

%% Interruzione di pagina
\newpage

\section{Tecnologie utilizzate durante lo sviluppo della tesi}
In questa sezione andremo ad esaminare le tecnologie che sono state utilizzate per realizzare la nuova versione del Passenger Counter.

%%TODO: Aggiungere nella bibliografia i riferimenti ai Datasheet delle telecamere
\subsection{Telecamere Intel RealSense}
Come dispositvi di acquisizione delle immagini sono state utilizzate le telecamere Intel RealSense. Esse sono telecamere ad infrarossi di nuova generazione che permettono la ricostruzione delle informazioni di profondit\`a della scena filmata sfruttando diverse tecniche di elaborazione delle immagini.

\subsubsection{Intel RealSense R200}

\begin{figure}[h!]
  \includegraphics[width=.49\textwidth]{R200.jpg}
  \includegraphics[width=.49\textwidth]{r200_module.png}
  \centering
  \caption{Telecamera R200}
  \label{fig:R200}
\end{figure}

%% TODO: Migliorare spiegazione funzionamento ricostruzione 3D stereoscopico. Scarica immagine che schematizzi funzionamento.
Il funzionamento di queste telecamere \`e lo stesso del Passenger Counter Eurotech. Sul modulo sono presenti due illuminatori ad infrarossi che illuminano la scena. Due telecamere stereoscopiche acquisiscono due immagini leggermente diverse dovute al diverso posizionamento delle telecamere. Per mezzo di algoritmi di image processing ricostruiscono l'informazione di profondit\`a dovuta alla visione della stessa scena da due punti di vista differenti. Per ottenere una ricostruzione delle informazioni in realtime \`e stato usato un circuito ASIC per raggiungere le prestazioni desiderate.

Specifiche tecniche:

\begin{center}
	\begin{tabular}{|c|c|c|}
	\hline
	& Depth Stream & Color Stream \\ \hline
	Risoluzione massima & 640 x 480 & 1920 x 1080 \\ \hline
    Frame rate massimo & 90fps & 60fps \\ \hline
    FOV (WxH) & 56 x 43 & 70 x 43 \\ \hline
    Indoor Range & 0.7 - 3.5m & - \\ \hline
    OutdoorRange & 10m & - \\
    \hline
	\end{tabular}
\end{center}

%% Interruzione di pagina
\newpage

\subsubsection{Intel RealSense SR300}

\begin{figure}[h!]
  \includegraphics[width=.49\textwidth]{SR300.jpg}
  \includegraphics[width=.49\textwidth]{SR300_module.png}
  \centering
  \caption{Telecamera SR300}
  \label{fig:SR300}
\end{figure}

La telecamera S300 utilizza una tecnica di rilevamento tridimensionale nota come luce strutturata. Il proiettore presente sul modulo proietta un pattern noto sulla scena. La deformazione dell'immagine proiettata permette ai sistemi di visione di calcolare la profondit\`a degli oggetti colpiti ed ottenere altre informazioni sulla superficio. L'acquisizione delle immagini a infrarossi in questo caso viene effettuata da un singolo scanner 3D a luce strutturata. In figura \ref{fig:SchemaFunzSR300} \`e schematizzato il funzionamento della telecamera.
Anche i questo caso l'informazione di profondit\`a viene ricostruita utilizzando un circuito ASIC.

\begin{figure}[h!]
  \includegraphics[width=\textwidth]{DepthDataFlowSR300.png}
  \centering
  \caption{Schema funzionamento Intel RealSense SR300}
  \label{fig:SchemaFunzSR300}
\end{figure}

%% Interruzione di pagina
\newpage

Specifiche tecniche:

\begin{center}
	\begin{tabular}{|c|c|c|}
	\hline
	& Depth Stream & Color Stream \\ \hline
	Risoluzione massima & 640 x 480 & 1920 x 1080 \\ \hline
    Frame rate massimo & 60fps & 60fps \\ \hline
    FOV (WxH) & 71 x 55  & 68 x 41 \\ \hline
    Indoor Range & 0.2 - 1.5m & - \\ \hline
    OutdoorRange & - & - \\
    \hline
	\end{tabular}
\end{center}

\subsubsection{Libreria e formato immagine}
Per l'interfacciamento software la Intel fornisce una libreria in C++ tramite la quale \`e possibile recuperare i frame dalle telecamere: la libreria \textbf{librealsense}\footnote{Repository libreria all'indirizzo: https://github.com/IntelRealSense/librealsense}. Essa provvede all'inizializzazione delle telecamere, apertura dei vari stream disponibili, impostazione di risoluzione e framerate. Gli stream disponibili sono: le immagini a colori, le immagini della telecamera a infrarossi e le immagini in cui \`e codificata l'informazione di profondit\`a. Nonostante usino tencologie diverse entrambe le telecamere forniscono lo stesso formato in uscita.

Lo stream di profondit\`a \`e uno stream di immagini a 16 bit privi di segno in cui il valore di ogni pixel rappresenta la distanza dalla telecamera. La distanza \`e fornita a meno di un fattore di scala reperibile per mezzo di una semplice chiamata di funzione. \`E possibile quindi risalire alla distanza in metri di un pixel dalla telecamera semplicemente moltiplicando il valore del pixel per il fattore di scala.
Vi \`e una eccezione a questa codifica in quanto, il valore del pixel 0, \`e attribuito ai pixel per i quali non \`e stato possibile ricavare informazione sulla distanza, ci\`o pu\`o essere dovuto a problemi di range o esposizione dell'immagine.

Vi sono per\`o delle minime differenze tra le due telecamere per quanto riguarda la risoluzione della distanza calcolata:
\begin{itemize}
\item Per la Intel RealSense SR300: il formato dello stream di profondit\`a \`e di 16 bit privi di segno interpolati su un range di 8m nonostante il range della telecamere arrivi ad un massimo di 1,5m. Ci\`o implica che la risoluzione della profondit\`a sia 0,125mm. 
\item Per la Intel RealSense R200: il formato \`e di 16 bit privi di segno interpolati su un range di circa 65m. Ci\`o implica una risoluzione di profondit\`a di 1mm.
\end{itemize}
Questo fatto non ha per\`o comportato problemi dal punto di vista della funzionalit\`a dell'applicazione in quanto la risoluzione e precisione delle telecamere non \`e critica per il corretto funzionamento del contatore.

%% Interruzione di pagina
\newpage

%% Copiato spudoratamente dalla mia presentazione su OpenCV.
\subsection{Libreria per l'image processing: OpenCV}\label{OpenCV}
Siccome una parte centrale della tesi verteva sull'elaborazione delle immagini \`e stato necessario integrare nel progetto l'uso della libreria OpenCV, ormai standard de facto nell'ambito dell'elaborazione delle immagini.

OpenCV, acronimo di Open Source Computer Vision, \`e una libreria software multipiattaforma finalizzata all'image processing real-time e alla computer vision. La libreria \`e rilasciata tramite licenza Berkeley Software Distribution (BSD) quindi \`e ad uso gratuito sia per fini accademici che commerciali. Contiene pi\`u di 2500 algoritmi pre-ottimizzati per le operazioni di image processing, comupter visione e machine learning pi\`u comuni. Sviluppata in C/C++, \`e dotata di interfacce verso C, Python, Java e MATLAB. Poich\`e finalizzata all'utilizzo real-time, la libreria sfrutta molteplici interfacce per l'accelerazione hardware (CUDA, OpenCL, Intel Integrated Perfomance Primitives). OpenCV ha una struttura modulare, il che significa che il pacchetto include diverse librerie statiche o condivise.

\subsubsection{Moduli principali e finalit\`a}
\begin{itemize}
\item \textbf{Modulo core}: funzionalit\`a di base. Lo scopo del modulo \`e  definire interfacce e funzionalit� che permettano di semplificare la manipolazione di immagini e flussi video. Il modulo contiene le funzionalit� di  base delle libreria e ne definisce le strutture fondamentali nonch\'e gestisce la memoria. 
\item \textbf{Modulo highui}: High-level GUI e Media I/O. l modulo HighGUI \`e stato progettato per fornire funzioni che permettano di provare le funzionalit\`a della libreria ed osservare i risultati velocemente. Fornisce semplici interfacce per creare e manipolare finestre che possano visualizzare immagini e aggiungere slider alle finestre, gestire semplici eventi come click del mouse e comandi da tastiera.
\item \textbf{Modulo imgproc}: image processing. Il modulo di Image Processing contiene funzioni e classi per la manipolazione di immagini. Ci\`o comprende:
    \begin{itemize}
    \item Image Filtering: funzioni per la convoluzione di immagini con un kernel, dilatazione di immagini, filtro Sobel, GaussianBlur ecc...
    \item Trasformazioni geometriche: ridimensionamenti, warping ecc...
    \item Funzioni per il disegno: permettono di disegnare semplici forme sulle immagini che vengono manipolate
    \item Funzioni per l'analisi delle immagini: istogrammi, analisi strutturale e descrittori di forme, motion analysis e tracking di oggetti.
    \end{itemize}
\item \textbf{Modulo videoio}: lettura e scrittura di file, nonch\`e analisi di video. Le funzioni principali implementate sono: analisi del movimento, sottrazione del background, rilevamento e tracciamento di oggetti.
\item \textbf{Modulo objdetect}: rilevazione di oggetti. Implementa funzioni e classi per il rilevamento e tracciamento di oggetti all'interno di immagini e flussi video. OpenCV ottiene tutto ci\`o facendo leva sui Haar Feature-based Cascade Classifier e Histogram of Oriented Gradients object detector.
\item \textbf{Modulo ml}: machine learning. La Machine Learning Library (MLL) \`e un insieme di classi e funzioni per la classificazione statistica, regressione e clustering dei dati. La maggior parte degli algoritmi di classificazione e regressione sono implementati come classi C++. Algoritmi implementati dal modulo:
    \begin{itemize}
    \item Artificial Neural Networks / Multi-Layer Perceptrons.
    \item Tree Classifier.
    \item Expectation Maximization algorithm.
    \item K-nearest Neighbors model.
    \item Logistic Regression.
    \item Normal Bayes Classifier.
    \item Support Vector Machines (SVM).
    \item Random forest predictor.
    \item Sochastic Gradient Descent SVM classifier.
    \end{itemize}
\end{itemize}

La struttura fondamentale della libreria \`e l'oggetto Mat, il quale \`e usato come contenitore delle immagini. Esso \`e una classe C++ formata da due componenti principali
\begin{itemize}
\item L'header: contenente informazioni come la dimensione della matrice, il metodo usato per salvarla, l'indirizzo dove \`e salvata ed un puntatore alla matrice.
\item La matrice: contenente i valori dei pixel dell'immagine la cui dimensionalit\`a dipende dal metodo utilizzato per salvare l'immagine (B/N, RGB ecc...)
\end{itemize}
Questa struttura \`e la stessa per tutti i tipi di immagine, indipendentemente dal formato nella quale \`e salvata. Questa astrazione facilita notevolmente la manipolazione delle immagini in quanto ci si riconduce velocemente ad una matrice di pixel manipolabile questa conversione \`e garantita dal modulo imgcodecs, la quale funzione non \`e altro che recuperare i dati sui pixel dell'immagine sulla quale si sta lavorando e generare i metadati per il tipo Mat.

\subsubsection{Storia}
Lanciato ufficialmente nel 1999, il progetto OpenCV era parte di una iniziativa Intel finalizzata all'avanzamento delle applicazioni CPU-intensive che includeva ray tracing real-time e display 3D. Inizialmente le finalit\`a del progetto erano descritte come:
\begin{itemize}
\item Permettere l'avanzamento della computer vision fornendo non solo una libreria di codice open-source ma anche pre ottimizzato per costruire l'infrastruttura base di applicazioni di computer visione.
\item Condividere conoscenza sulla computer vision fornendo una infrastruttura comune sulla quale tutti gli sviluppatori potessero costruire.
\item Far avanzare le applicazioni commerciali basate sulla computer vision rendendo il codice portabile, ottimizzato per le performance.
\end{itemize}
Ad oggi il supporto al progetto \`e dato dall'organizzazione no-profit OpenCV.org, la quale mantiene uno sviluppatore e un sito per la documentazione. La libreria \`e ormai lo standard de facto per le applicazioni di computer vision e image processing, vista la sua diffusione e la community molto attiva.

Per lo sviluppo dell'applicazione di Passenger Counter la scelta di utilizzare questa libreria \`e stata quasi obbligata visto l'ottimo supporto e diffusione della libreria. Inoltre la vasta portabilit\`a e l'attenzione alle performance la rendono particolarmente adatto al caso embedded real-time.

%% Interruzione di pagina
\newpage

%% TODO: Rivedi cosa hai scritto e amplia ancora se riesci
\subsection{Ambiente di sviluppo per l'OS: Yocto Project}\label{Yocto}
Per integrare nella piattaforma target gli strumenti necessari al funzionamento dell'applicativo \`e stato necessario realizzare una distribuzione Linux customizzata adatta alle nostre esigenze. Per farlo \`e stato utilizzato il progetto Yocto, standard industriale de facto per realizzare distribuzioni Linux embedded.

Yocto Project \`e un insieme di strumenti open source finalizzati alla creazione di distribuzioni Linux per sistemi embedded, corredate da toolchain di cross-compilazione ed emulatori, che siano indipendenti dall'architettura hardware. Yocto Project \`e quindi un ombrello sotto il quale si raccolgono vari sotto-progetti volti allo sviluppo di sistemi Linux embedded.

Le architettura supportate dal progetto sono le pi\`u diffuse nell'ambito embedded: ARM, MIPS, PowerPC e x86/x86-64. Una parte fondamentale di questo progetto \`e il build system open source basato sull'architettura OpenEmbedded. Questa implementazione di OpenEmbedded \`e chiamata Poky.

OpenEmbedded \`e framework software usato per creare distribuzioni Linux embedded. Il build system \`e basato sulle ricette BitBake le quali sono degli script bash specializzati che automatizzano la compilazione e la installazione di pacchetti software. Le ricette BitBake consistono di un URL che punta alla sorgente del pacchetto software, dipendenze e opzioni di compilazione ed installazione. Durante il processo di build sono usate per tenere traccia delle dipendenze ed effettuare cross-compilazione dei pacchetti affinch\`e sia possibile installarli sulla target distribution. BitBake stesso \`e uno dei componenti fondamentali di Yocto Project: \`e un build automation software, ovvero un tool che legge dei metadati sottoforma di "ricette" ed esegue task in base a dette ricette.

\subsubsection{Componenti del progetto Yocto}
\begin{itemize}
\item OpenEmbedded: \`e un build framework per embedded Linux. Permette la creazione di distribuzioni Linux complete per sistemi immersi e offre un ambiente di cross-compilazione completo. Il build system \`e basato su ricette BitBake.
\item BitBake: \`e un build automation software, ovvero un tool che legge dei metadati sottoforma di "ricette" ed esegue task in base a dette ricette. \`e uno dei componenti fondamentali di Yocto Project.
\item Poky: \`e una reference distribution di Yocto Project. Contiene l'OpenEmbedded Build System(BitBake e OpenEmbedded) assieme a un insieme di metadati che permettono di cominciare a costruire la propria distribuzione Linux embedded customizzata.
\item Layers: il build system del progetto Yocto \`e composto da layer. I layer sono una collezione logica di ricette che rappresentano stack di applicazioni o Board Support Packaged (BSP). I BSP contengono i pacchetti e i driver fondamentali necessari per costruire una distribuzione Linux per una specifica board o architettura. Solitamente questi BSP sono mantenuti dai produttori dell'hardware e sono l'interfaccia tra l'OS e l'hardware che lo esegue.
\end{itemize}

%% Interruzione di pagina
\newpage

\begin{figure}[h!]
  \includegraphics[width=\textwidth]{yocto-environment.png}
  \centering
  \caption{Workflow progetto Yocto}
  \label{fig:WorkflowYocto}
\end{figure}

\subsubsection{Processo di build di una distribuzione}
Dopo aver configurato e lanciato il processo di build il build system provvede ad eseguire le seguenti operazioni per ogni pacchetto e applicazione che si desidera installare:
\begin{enumerate}
\item Recupera autonomamente i sorgenti remoti e locali.
\item Procede alla cross-compilazione dei sorgenti.
\item Genera i file .deb/.rpm/.ipk dipendentemente dalla configurazione.
\end{enumerate}
Una volta che sono stati generati tutti i pacchetti richiesti procede alla generazione dell'immagine della distribuzione, genera quindi il filesystem, installa i pacchetti e genera l'immagine che pu\`o essere eseguita sulla macchina target. In seguito \`e possibile generare la toolchain per la cross-compilazione di sorgenti per l'immagine della macchina target, questo processo generer\`a un piccolo installer il quale installer\`a la toolchain nella macchina dedicata allo sviluppo delle applicazioni per la piattaforma target.

%% Interruzione di pagina
\newpage

Per lo sviluppo dell'applicazione del Passenger Counter \`e stato necessario realizzare l'immagine della distribuzione Linux opportunamente modificata e la toolchain di cross-sviluppo prima di cominciare lo sviluppo dell'applicazione. Una volta completati questi passaggi si \`e potuti passare allo sviluppo dell'applicazione sulla SDKMachine. Siccome si \`e passati attraverso diverse tecnologie il processo \`e stato ripetuto pi\`u volte per poter integrare le nuove tecnologie all'interno della toolchain.

\begin{figure}[h!]
  \includegraphics[scale=1]{cross-development-toolchains.png}
  \centering
  \caption{Toolchain di cross-sviluppo del progetto Yocto}
  \label{fig:ToolchainYocto}
\end{figure}

%% Interruzione di pagina
\newpage

\subsection{Framework OSGi}\label{OSGi}
In pausa finch\`e non sei sicuro di usarlo.

% Capitolo 3: Corpo della tesi. Sviluppo del progetto: cosa ho fatto e come l'ho fatto.
\chapter{Sviluppo del progetto}
In questo capitolo verr\`a riportato come \`e stato affrontato lo sviluppo dell'applicazione e come sono stati affrontati i problemi riscontrati.

\section{Analisi degli ambienti di sviluppo disponibili per l'image processing embedded}
Nella fase iniziale del progetto si \`e studiato quale potesse essere l'ambiente di sviluppo adatto all'applicazione. Sono state considerate principalmente due possibilit\`a analizzate nei paragrafi seguenti.

% TODO: Approfondire di pi\`u?
\subsection{Ambiente di sviluppo Xilinx}
Inizialmente si era pensato di sfruttare i tool di sintesi di alto livello messi a disposizione da Xilinx per lo sviluppo in quanto sarebbe stato possibile realizzare applicazioni ad altissime prestazioni con la flessibilit\`a di un linguaggio ad alto livello come C/C++. Uno dei target di punta di questi sistemi di sviluppo \`e appunto l'image processing e come supporto forniscono anche librerie proprietarie pre-ottimizzate per l'hardware Xilinx.

\subsubsection{Hardware}
La piattaforma target sarebbe stata una Avnet Zedboard accompagnata dal modulo di espansione per aggiungere una HDMI in input. Specifiche tecniche:
\begin{itemize}
\item ZYNQ-7000 SOC XC7Z020-CLG484-1: Dual ARM Cortex-A9 MPCore Up to 667 MHz operation NEON Processing / FPU Engines
\item Memoria: 512 MB DDR3 256 Mb Quad-SPI Flash Full size SD/MMC card cage 4 GB SD card
\item Connessioni: 10/100/1000 Ethernet, USB OTG (Device/Host/OTG), USB UART
\item Espanzione: FMC (Low Pin Count) (5) Pmod headers (2x6)
\item Video: HDMI output (1080p60 + audio), VGA connector, 128 x 32 OLED
\end{itemize}

%% Interruzione di pagina
\newpage

In figura \ref{fig:Zynq7000} \`e riportato il diagramma del SoC Zynq-7000.

\begin{figure}[h!]
  \includegraphics[width=\textwidth]{Zynq7000.png}
  \centering
  \caption{Diagramma del SoC Zynq-7000}
  \label{fig:Zynq7000}
\end{figure}

%% Interruzione di pagina
\newpage

\subsubsection{Piattaforma per lo sviluppo SDSoC}
La piattaforma per lo sviluppo SDSoC permette uno sviluppo integrato dell'applicativo e dell'IP hardware utilizzando un linguaggio di alto livello come il C o il C++. Vista la maggiore familiarit\`a con i linguaggi di alto livello l'utilizzo di questo ambiente di sviluppo avrebbe comportato una maggior flessibilit\`a e velocit\`a nello sviluppo con il vantaggio di potersi avvalere dell'accelerazione hardware sfruttando la FPGA presente sul SoC.

\begin{figure}[h!]
  \includegraphics[scale=1]{XilinxSDSoC.png}
  \centering
  \caption{Schematizzazione workflow SDSoC}
  \label{fig:SDSoC}
\end{figure}

Come ulteriore punto a favore di questo ambiente, Xilinx vantava la compatibilit\`a del sistema con OpenCV per l'esecuzione su CPU mentre rendeva disponibile una libreria compatibile con OpenCV per l'esecuzione su FPGA. Quest'ultima prende il nome di \textbf{AuvizCV}, e non \`e altro che un subset delle funzioni disponibili in OpenCV pre-ottimizzate per l'esecuzione sulle FPGA Xilinx.

%TODO: Completare
\subsection{Ambiente di sviluppo x86 Intel - Yocto}
Come seconda possibilit\`a si \`e considerata la possibilit\`a di usare l'hardware reso disponibile da Eurotech e realizzare da zero l'infrastruttura software sfruttando il progetto Yocto.

\subsubsection{Hardware}
L'azienda ha fornito come piattaforma hardware un ReliGate 20-25.

\subsection{Motivazioni per la scelta finale}
Si � infine scelto di operare all'interno del secondo sistema di sviluppo per molteplici motivi.

%% Fine dei capitoli normali, inizio dei capitoli-appendice (opzionali)
\appendix

\part{Appendici}

\chapter{Altro capitolo}
Sed purus libero, vestibulum ut nibh vitae, mollis ultricies augue. Pellentesque velit libero, tempor sed pulvinar non, fermentum eu leo. Duis posuere eleifend nulla eget sagittis. Nam laoreet accumsan rutrum. Interdum et malesuada fames ac ante ipsum primis in faucibus. Curabitur eget libero quis leo porttitor vehicula eget nec odio. Proin euismod interdum ligula non ultricies. Maecenas sit amet accumsan sapien.

%% Parte conclusiva del documento; tipicamente per riassunto, bibliografia e/o indice analitico.
\backmatter

%% Riassunto (opzionale)
\summary
Maecenas tempor elit sed arcu commodo, dapibus sagittis leo egestas. Praesent at ultrices urna. Integer et nibh in augue mollis facilisis sit amet eget magna. Fusce at porttitor sapien. Phasellus imperdiet, felis et molestie vulputate, mauris sapien tincidunt justo, in lacinia velit nisi nec ipsum. Duis elementum pharetra lorem, ut pellentesque nulla congue et. Sed eu venenatis tellus, pharetra cursus felis. Sed et luctus nunc. Aenean commodo, neque a aliquam bibendum, mauris augue fringilla justo, et scelerisque odio mi sit amet diam. Nulla at placerat nibh, nec rutrum urna. Donec ut egestas magna. Aliquam erat volutpat. Phasellus vestibulum justo sed purus mattis, vitae lacinia magna viverra. Nulla rutrum diam dui, vel semper mi mattis ac. Vestibulum ante ipsum primis in faucibus orci luctus et ultrices posuere cubilia Curae; Donec id vestibulum lectus, eget tristique est.

%% Bibliografia (opzionale)
\bibliographystyle{plain_\languagename}%% Carica l'omonimo file .bst, dove \languagename � la lingua attiva.
%% Nel caso in cui si usi un file .bib (consigliato)
\bibliography{thud}
%% Nel caso di bibliografia manuale, usare l'environment thebibliography.

%% Per l'indice analitico, usare il pacchetto makeidx (o analogo).

\end{document}

--- Istruzioni per l'aggiunta di nuove lingue ---
Per ogni nuova lingua utilizzata aggiungere nel preambolo il seguente spezzone:
    \addto\captionsitalian{%
        \def\abstractname{Sommario}%
        \def\acknowledgementsname{Ringraziamenti}%
        \def\authorcontactsname{Contatti dell'autore}%
        \def\candidatename{Candidato}%
        \def\chairname{Direttore}%
        \def\conclusionsname{Conclusioni}%
        \def\cosupervisorname{Co-relatore}%
        \def\cosupervisorsname{Co-relatori}%
        \def\cyclename{Ciclo}%
        \def\datename{Anno accademico}%
        \def\indexname{Indice analitico}%
        \def\institutecontactsname{Contatti dell'Istituto}%
        \def\introductionname{Introduzione}%
        \def\prefacename{Prefazione}%
        \def\reviewername{Controrelatore}%
        \def\reviewersname{Controrelatori}%
        %% Anno accademico
        \def\shortdatename{A.A.}%
        \def\summaryname{Riassunto}%
        \def\supervisorname{Relatore}%
        \def\supervisorsname{Relatori}%
        \def\thesisname{Tesi di \expandafter\ifcase\csname thud@target\endcsname Laurea\or Laurea Magistrale\or Dottorato\fi}%
        \def\tutorname{Tutor aziendale%
        \def\tutorsname{Tutor aziendali}%
    }
sostituendo a "italian" (nella 1a riga) il nome della lingua e traducendo le varie voci.
